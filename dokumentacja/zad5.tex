\documentclass[]{article}

%opening
\title{}
\author{}
\date{}

\usepackage[utf8]{inputenc}
\usepackage[T1]{fontenc}

\begin{document}

\maketitle

\section{Testowa konstrukcja}

\section{Porównanie demonstratora z pomiarem wzorcowym}
Scheamt blokowy z akcelerometrem, pomiar laserowy. Czujnik laserowy zbiera położenie, a czujnik tensometryczny zbiera przyspieszenie. Dokonano dwukrotnego zróżniczkowania położenia

Etap 1: Pomiar w osi czujnika. schemat blokowy zdj telefon. Korelacja dwóch sygnałów i wykres w czasie 



Potwierdzenie poprawności funkcjonowania układu w zakresie przetwarzania wielkości mechanicznych (przemieszenia, przyspieszenia) na wielkości elektryczne. 

Wizualizacja wyników w formie wykresów czasowych oraz wizualizacji 3d

\section{Pomiar w innym kierunku niż orientacja wybranego czujnika układu pomiarowego}

[0,25; 0,25; 1]

schemat ze zdj


\section{}
Układ sprzętowy wielokanałowego synchronicznego rejestratora drgań osadzony zostanie na testowej
konstrukcji, która pobudzana będzie do drgań. Konstukcja opomiarowana zostanie w sposób umożliwający
dokonanie pomiarów metodami pozwalającymi na budowę modelu wzorcowego. Następnie zostanie
przeprowadzony pomiar z użyciem opracowanego rozwiązania, po czym wyniki obu pomiarów (wzorcowego
oraz dokonanego wytworzonym demonstratorem) zostaną porównane. Na tym etapie potwierdzona zostanie
poprawność funkcjonowania układu w zakresie przetwarzania wielkości mechanicznych (przemieszczenia,
przyśpieszenia) na wielkości elektryczne. W dalaszej kolejności rejestrowane dane przekazane będą za
pośrednictwem opracowanego interfejsu komunikacyjnego i protokołów wymiany danych do modułu
oprogramowania analizującego.
Nastąpi w tej fazie oszacowanie poprawności funkcjonwania na poziomie transmisji danch oraz
wizualizacji wyników w formie wykresów czasowych oraz wizualizacji 3d. Ostatnim etapem testów będzie
porównanie wyników obliczeń dokonanych za pomocą opracowanego modelu matematycznego z wynikami
pomiarów wzorcowych. W tym celu dokonany zostanie pomiar wzorcowy amplitudy i częstotliwości drgać w
kierunku innym niż orientacja wybranego czujnika ukłądu pomiarowego, po czym nastąpi porównanie
zarejestrowanych danych wzorcowych z danymi przetworzonymi przez model matematyczny.

W efekcie realizacji zadania powstanie demonstrator kompletnego systemu monitoringu i analizy drgań
oraz zestwienie wyników pomiarów empirycznych oraz porównań  wyników obliczeń wg modelu
matematycznego z pomiarami wzorcowymi.
\end{document}
